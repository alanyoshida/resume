%%%%%%%%%%%%%%%%%
% This is an example CV created using altacv.cls (v1.1, 21 November 2016) written by
% LianTze Lim (liantze@gmail.com), based on the 
% Cv created by BusinessInsider at http://www.businessinsider.my/a-sample-resume-for-marissa-mayer-2016-7/?r=US&IR=T
% 
%% It may be distributed and/or modified under the
%% conditions of the LaTeX Project Public License, either version 1.3
%% of this license or (at your option) any later version.
%% The latest version of this license is in
%%    http://www.latex-project.org/lppl.txt
%% and version 1.3 or later is part of all distributions of LaTeX
%% version 2003/12/01 or later.
%%%%%%%%%%%%%%%%

%% If you want to use \orcid or the
%% academicons icons, add "academicons"
%% to the \documentclass options. 
%% Then compile with XeLaTeX or LuaLaTeX.
% \documentclass[10pt,a4paper,academicons]{altacv}
\documentclass[10pt,a4paper]{altacv}

%% AltaCV uses the fontawesome and academicon fonts
%% and packages. 
%% See texdoc.net/pkg/fontawecome and http://texdoc.net/pkg/academicons for full list of symbols.
%% When using the "academicons" option,
%% Compile with LuaLaTeX for best results. If you
%% want to use XeLaTeX, you may need to install
%% Academicons.ttf in your operating system's font %% folder.


% Change the page layout if you need to
\geometry{left=1cm,right=9cm,marginparwidth=6.8cm,marginparsep=1.2cm,top=1cm,bottom=1cm}

% Change the font if you want to.

% If using pdflatex:
\usepackage[utf8]{inputenc}
\usepackage[T1]{fontenc}
\usepackage[default]{lato}

% If using xelatex or lualatex:
% \setmainfont{Lato}

% Change the colours if you want to
\definecolor{VividPurple}{HTML}{3E0097}
\definecolor{SlateGrey}{HTML}{2E2E2E}
\definecolor{LightGrey}{HTML}{666666}
\colorlet{heading}{VividPurple}
\colorlet{accent}{VividPurple}
\colorlet{emphasis}{SlateGrey}
\colorlet{body}{LightGrey}

% Change the bullets for itemize and rating marker
% for \cvskill if you want to
\renewcommand{\itemmarker}{{\small\textbullet}}
\renewcommand{\ratingmarker}{\faCircle}

\begin{document}
\name{Alan Luis Yoshida de Oliveira}
\tagline{DevSecOps Engineer}
\personalinfo{%
  % Not all of these are required!
  % You can add your own with \printinfo{symbol}{detail}
  \email{alanyoshida88@gmail.com}
%   \phone{+49-173-6895039}
  \location{São Paulo, Brazil}
  \linkedin{linkedin.com/in/alanyoshida88}
  \twitter{@alanyoshida}
  \github{github.com/alanyoshida}
  
%   \github{github.com/mmayer} % I'm just making this up though.
%   \orcid{orcid.org/0000-0000-0000-0000} % Obviously making this up too. If you want to use this field (and also other academicons symbols), add "academicons" option to \documentclass{altacv}
}

%% Make the header extend all the way to the right, if you want. Extend the right margin by 8cm (=6.8cm marginparwidth + 1.2cm marginparsep)
\begin{adjustwidth}{}{-8cm}
\makecvheader
\end{adjustwidth}

%% Provide the file name containing the sidebar contents as an optional parameter to \cvsection.
%% You can always just use \marginpar{...} if you do
%% not need to align the top of the contents to any
%% \cvsection title in the "main" bar.
\cvsection[page1sidebar]{Experience}

\cvevent{SRE - Infraestructure team}{Dafiti (Global Fashion Group)}{May 2019 -- Ongoing}{São Paulo, Brazil}
Dafiti is one of the biggest fashion ecommerces in latim america.
\linebreak
\begin{itemize}
\item Responsible for improving the CI/CD of projects and to bring tools and automations for improving the developers life.
\item Implemented ArgoCD + Argo Workflows and Eks pci compliant kubernetes cluster.
\item Automated Black Friday Stress test and acceptance tests using argo workflows.
\item Applying automatically manifests for cluster tooling and microservices deployments.
\item Creating and helping with implementing a RFC model for improving the company communication and development of patterns and standards.
\item Writing and evolving a maturity model documentation for better classification of the maturity of microservices.
\end{itemize}

\divider

\cvevent{Software Engineer - Liquidation team}{PagSeguro}{August 2006 -- September 2018}{São Paulo, Brazil}
Pagseguro is a company focused mainly in financial transactions as credit card transactions using web technologies and also a credit/debit card machine. It's a big player under the UOL Group.
\linebreak

\begin{itemize}
\item Payment Schedule and Liquidation Using the following tecnologies: Spring boot, H2, Flyway, Swagger, AWS SQS, SNS, Lambda, Cloudwatch, Lombok, JUnit 5, Jenkins, Mesos
\end{itemize}

\divider

\cvevent{Software Engineer - Pabx Team}{Locaweb}{Aug 2016 -- September 2018}{São Paulo, Brazil}
Locaweb is the largest hosting company in brazil, with about one thousand employees. They sell IAAS (Infrastructure as a service), SAAS (Software as a service), PAAS (Product as a service) with many fronts of business including hosting, voip, data storages, vps and more.
\linebreak
\begin{itemize}
\item I had the responsibility to develop and maintain the legacy code and maintain the project devops environment with Jenkins, Docker, Gitlab CI, Mesos + Marathon.
\item Development with Java with Vert.x Framework, GraphQL (java-graphql, graphql-spqr), VueJs (Vue router, Vuex, ES6, webpack, karma, mocha, chai, sinon), Websockets, Gitlab CI ( Build and deploy to mesos container), Mesos + Marathon config to build the container environment, Lombok

\end{itemize}
\divider

\clearpage


\end{document}
